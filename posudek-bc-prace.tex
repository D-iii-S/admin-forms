%
% Sablona pro psani posudku bakalarskych praci (oboru Informatika na MFF UK)
% ==========================================================================
%
% Tento soubor je siren pod licenci Apache, verze 2.0 (viz soubor LICENSE).
% This file is distributed under Apache License, ver. 2.0 (see file LICENSE).
%
% Aktualni verzi tohoto souboru najdete na teto webove strance:
% https://github.com/D-iii-S/admin-forms
% 
% Tento soubor je pripraven v kodovani UTF-8. V pripade, ze vas system
% toto kodovani nepodporuje, muzete jej zmenit v ramci nacitani balicku
% inputenc nize. Nezapomente ale v takovem pripade zkontrolovat, ze se
% v poradku vysazi vsechny polozky s hacky a carky (komentare jsou z
% duvodu archaicke kompatibility zatim udrzovany bez hacku a carek).
%
% Tato sablona prepoklada, ze autor posudku zna rozumne system LaTeX a
% proto vyplnovani posudku probiha formou rozsirovani vlastniho tela
% dokumentu, nikoliv definici maker pro jednotlive casti jako u jinych
% podobnych sablon. Jednotlive casti jsou komentovany primo v TeXovem
% kodu, sablona jako takova je pak schopna graficky zvyraznit chybejici
% udaje i ve vyslednem PDF souboru.
%
% Pred finalnim odevzdanim posudku muze byt vhodne rucne upravit
% zalamovani stranek. Interni pouziti prostredi longtable zarucuje
% rozumne chovani ve vetsine pripadu; preciznejsi kontrolu nad mistem
% zalomeni (typicky pred zacatkem jednotlivych bloku) lze vynutit
% napr. vlozenim prikazu \pagebreak.
%
% V prostredi evaluation je tez k dispozici prikaz \evalbreak
%
% Pokud nechcete menit vzhled dokumentu, preskocte nyni na zacatek tela
% dokumentu, tj. ke znacce \begin{document}.
%
\documentclass[12pt,a4paper]{article}

\usepackage[utf8x]{inputenc}

% Volba jazyka formulare, CZ nebo EN.
\usepackage[CZ]{optional}
% Ceske deleni slov
\usepackage[czech]{babel}

\usepackage[top=2cm,left=2cm,right=2cm,bottom=2.5cm]{geometry}

\usepackage{ifthen}
\usepackage{xcolor}
\usepackage{xstring}
\usepackage{setspace}
\usepackage{longtable}

\usepackage{array}
\def\arraystretch{1.1}%

\newlength{\myparindent}
\setlength{\myparindent}{1em}

% Vlastni typy, pro P definujeme odsazeni odstavce.
\newcolumntype{C}[1]{>{\centering\let\newline\\\arraybackslash\hspace{0pt}}m{#1}}
\newcolumntype{R}[1]{>{\raggedleft\let\newline\\\arraybackslash\hspace{0pt}}m{#1}}
\newcolumntype{P}[1]{>{%
\let\newline\\\arraybackslash\hspace{\myparindent}%
\setlength{\parindent}{\myparindent}%
\renewcommand\noindent{\hspace{-\myparindent}}}p{#1}
}

% Zobrazi puvodni misto pouziti makra (radek) ve zdrojovem dokumentu.
\newcommand\showPosition{{\textcolor{red}{\footnotesize(\opt{CZ}{řádek}\opt{EN}{line}~\the\inputlineno)}}}

% Zobrazi dany text cervene.
\newcommand\errorText[1]{{\color{red} \textbf{#1}}}

% Vypise krizek, pokud alespon jedna znamka (parametr 2) je stejna jako parametr #1 (cislo policka)
\newcommand\makeMark[2]{\IfSubStr{#2}{#1}{X}{~}}

% Polozka v hlavicce.
% \field{Jmeno pole}{Hodnota}
% Pokud hodnota zacina na !, je pole vysviceno jako chybne.
\newcommand{\field}[2]{
\StrLeft{#2}{1}[\first]
\IfStrEq{\first}{!}{\errorText{#1} & \showPosition}{\textbf{#1}~~ & #2} \\
}

% Hlavicka prace.
% Predpokladame, ze pouze polozky \field budou uvnitr tohoto prostredi.
\newenvironment{topmatter}{%
\begin{center}
\opt{CZ}{\Large Posudek bakalářské práce}
\opt{EN}{\Large Bachelor Thesis Review}

\vspace{3mm}

\opt{CZ}{\large Matematicko-fyzikální fakulta Univerzity Karlovy}
\opt{EN}{\large Faculty of Mathematics and Physics, Charles University}
\end{center}

\renewcommand\medskip{\\}
\begin{tabular}{R{4cm}p{9cm}}
}{%
\end{tabular}

\vspace{2em}
}


% Vlozi dilci hodnoceni.
% \grade{Nazev}{Znamka jako cislo 1-4}{Detailnejsi popis}
\newcommand{\grade}[3]{
\hspace{-\parindent}%
\hspace{-3mm}
\begin{tabular}{p{11.1cm}|C{.8cm}|C{.8cm}|C{.8cm}|C{1.8cm}}
    \ifthenelse{\equal{#2}{!}}{\color{red} \textbf{#1}}{#1}
        \hfill
        \textit{\scriptsize \ifthenelse{\equal{#3}{}}{}{~\ldots$\,$#3}}
    & \makeMark{1}{#2}
    & \makeMark{2}{#2}
    & \makeMark{3}{#2}
    & \ifthenelse{\equal{#2}{!}}{\showPosition}{\makeMark{4}{#2}}
    \\
\end{tabular}
\\
\hline%
}


% Blok s hodnocenim.
\newenvironment{evaluation}[1]{%
\noindent
\begin{tabular}{p{11.1cm}C{.8cm}C{.8cm}C{.8cm}C{1.8cm}}
\hspace{-2ex} \textbf{#1} & \footnotesize \opt{CZ}{lepší}\opt{EN}{good} & \footnotesize OK & \footnotesize \opt{CZ}{horší}\opt{EN}{poor} & \footnotesize \opt{CZ}{nevyhovuje}\opt{EN}{insufficient}
\end{tabular}
\vspace{-1.5em}
\begin{longtable}{|P{17.2cm}|}
\hline%
}{%
\\
\hline
\end{longtable}
}

% Vytvoreni mista pro zalomeni stranky
\newcommand\evalbreak{\\\noindent{\small\emph{(pokračování na další straně)}}\\\hline}


% Zaverecna znamka.
% \finalgrade{Znamka slovne}{Ano/Ne - doporuceni k zvlastnimu oceneni}
\newcommand{\finalgrade}[2]{
\begin{tabular}{R{8cm}p{4cm}}
\opt{CZ}{\field{Celkové hodnocení}{#1}}
\opt{EN}{\field{Overall grade}{#1}}
\opt{CZ}{\field{Práci navrhuji na zvláštní ocenění}{#2}}
\opt{EN}{\field{Award level thesis}{#2}}
\end{tabular}
}


%%%%%%%%%%%%%%%%%%%%%%%%%%%%%%%%%%%%%%%%%%%%%%%%%%%%%%%%%%%%%%%%%
%                                                               %
%   ZDE zacnete vyplnovat vlastni hodnoceni bakalarske prace.   %
%                                                               %
%%%%%%%%%%%%%%%%%%%%%%%%%%%%%%%%%%%%%%%%%%%%%%%%%%%%%%%%%%%%%%%%%
%
% Prosím vyplňte hodnocení křížkem u každého kritéria.
% Hodnocení OK označuje práci, která kritérium vhodným způsobem splňuje.
% Hodnocení lepší a horší označují splnění nad a pod rámec obvyklý pro bakalářskou práci.
% Hodnocení nevyhovuje označuje práci, která by neměla být obhájena.
% Hodnocení v případě potřeby doplňte komentářem.
%
% Komentář prosím doplňte všude, kde je hodnocení jiné než OK.
%
\begin{document}

% Pouze pokud bude mit vase hodnoceni vice jak 2 stranky, ma smysl zapinat cislovani stranek.
\pagestyle{empty}


%
% Sazba hlavicky posudku
% ----------------------
%
% Vyplnte prislusne udaje misto hodnot ! v druhem argumentu makra \field.
%
\begin{topmatter}

% Informace o autorovi a jeho studiu
\field{\opt{CZ}{Autor práce}\opt{EN}{Thesis author}}{!}
\field{\opt{CZ}{Název práce}\opt{EN}{Thesis title}}{!}
\field{\opt{CZ}{Rok odevzdání}\opt{EN}{Year submitted}}{!}
\field{\opt{CZ}{Studijní program}\opt{EN}{Study program}}{\opt{CZ}{Informatika}\opt{EN}{Computer Science}}
\field{\opt{CZ}{Studijní obor}\opt{EN}{Study branch}}{!}

\medskip


\field{\opt{CZ}{Autor posudku}\opt{EN}{Review author}}{! \hfill \opt{CZ}{Vedoucí}\opt{EN}{Advisor}}
%\field{\opt{CZ}{Autor posudku}\opt{EN}{Review author}}{! \hfill \opt{CZ}{Oponent}\opt{EN}{Reviewer}}
\field{\opt{CZ}{Pracoviště}\opt{EN}{Department}}{!}
%\field{\opt{CZ}{Pracoviště}\opt{EN}{Department}}{Informatický ústav Univerzity Karlovy}
%\field{\opt{CZ}{Pracoviště}\opt{EN}{Department}}{Katedra aplikované matematiky}
%\field{\opt{CZ}{Pracoviště}\opt{EN}{Department}}{Katedra distribuovaných a spolehlivých systémů}
%\field{\opt{CZ}{Pracoviště}\opt{EN}{Department}}{Katedra softwarového inženýrství}
%\field{\opt{CZ}{Pracoviště}\opt{EN}{Department}}{Katedra softwaru a výuky informatiky}
%\field{\opt{CZ}{Pracoviště}\opt{EN}{Department}}{Katedra teoretické informatiky a matematické logiky}
%\field{\opt{CZ}{Pracoviště}\opt{EN}{Department}}{Středisko informatické sítě a laboratoří}
%\field{\opt{CZ}{Pracoviště}\opt{EN}{Department}}{Ústav formální a aplikované lingvistiky}

\end{topmatter}


%
% Souhrnne hodnocení cele prace
% -----------------------------
%
% Jednotliva hodnocení vyplnte jako cisla 1-4 v druhem argumentu makra \grade (misto znaku !).
% 1 oznacuje splneni kriteria nad urovni beznou pro bakalarske prace,
% 2 splneni na obvykle urovni,
% 3 splneni pod beznou urovni,
% 4 oznacuje zavazny nedostatek.
%
% Je mozne vyplnit misto jednoho cisla i vice cisel (napr. '12', v takovem pripade je znamka mezi lepsi a OK)
%
\begin{evaluation}{\opt{CZ}{K celé práci}\opt{EN}{Overall}}
\grade{\opt{CZ}{Obtížnost zadání}\opt{EN}{Assignment difficulty}}{!}{}
\grade{\opt{CZ}{Splnění zadání}\opt{EN}{Assignment fulfilled}}{!}{}
\grade{\opt{CZ}{Rozsah práce}\opt{EN}{Total size}}{!}{\opt{CZ}{textová i implementační část, zohlednění náročnosti}\opt{EN}{text and code, overall workload}}

% Do tohoto mista (pred konec prostredi evaluation) vlozte slovni komentar.
\end{evaluation}


%
% Hodnoceni textove casti prace
% -----------------------------
%
% Instrukce viz vyse.
%
\begin{evaluation}{\opt{CZ}{Textová část práce}\opt{EN}{Thesis Text}}
\grade{\opt{CZ}{Formální úprava}\opt{EN}{Form}}{!}{\opt{CZ}{jazyková úroveň, typografická úroveň, citace}\opt{EN}{language, typography, references}}
\grade{\opt{CZ}{Struktura textu}\opt{EN}{Structure}}{!}{\opt{CZ}{kontext, cíle, analýza, návrh, vyhodnocení, úroveň detailu}\opt{EN}{context, goals, analysis, design, evaluation, level of detail}}
\grade{\opt{CZ}{Analýza}\opt{EN}{Problem analysis}}{!}{\opt{CZ}{}\opt{EN}{}}
\grade{\opt{CZ}{Vývojová dokumentace}\opt{EN}{Developer documentation}}{!}{\opt{CZ}{}\opt{EN}{}}
\grade{\opt{CZ}{Uživatelská dokumentace}\opt{EN}{User Documentation}}{!}{\opt{CZ}{}\opt{EN}{}}

% Slovni komentar.
\end{evaluation}


%
% Hodnoceni implementacni casti prace
% -----------------------------------
%
% Instrukce viz vyse.
%
\begin{evaluation}{\opt{CZ}{Implementační část práce}\opt{EN}{Thesis Code}}
\grade{\opt{CZ}{Kvalita návrhu}\opt{EN}{Design}}{!}{\opt{CZ}{architektura, struktury a algoritmy, použité technologie}\opt{EN}{architecture, algorithms, data structures, used technologies}}
\grade{\opt{CZ}{Kvalita zpracování}\opt{EN}{Implementation}}{!}{\opt{CZ}{jmenné konvence, formátování, komentáře, testování}\opt{EN}{naming conventions, formatting, comments, testing}}
\grade{\opt{CZ}{Stabilita implementace}\opt{EN}{Stability}}{!}{\opt{CZ}{}\opt{EN}{}}

% Slovni komentar.
\end{evaluation}


%
% Zaverecne hodnoceni - znamka
% ----------------------------
%
% Prvni radek (se znamkou !) slouzi pouze pro upozorneni ve vystupnim
% souboru. Po vybrani znamky jej smazte nebo zakomentujte. Ke znamce
% je mozne pripsat formulace jako "spíše lepší" či "spíše horší".
%
% Druhy parametr makra (Ano/Ne) se tyka navrzeni teto prace na zvlastni oceneni.
%
\finalgrade{!}{!}
%\finalgrade{Výborně}{Ano}
%\finalgrade{Výborně}{Ne}
%\finalgrade{Velmi dobře}{Ne}
%\finalgrade{Dobře}{Ne}
%\finalgrade{Neprospěl(a)}{Ne}
%\finalgrade{Excellent}{Yes}
%\finalgrade{Excellent}{No}
%\finalgrade{Very Good}{No}
%\finalgrade{Good}{No}
%\finalgrade{Failed}{No}


%
% Datum odevzdani posudku
% -----------------------
%
\vspace{2em}
\opt{CZ}{Datum \hfill Podpis \hspace{5cm}}
\opt{EN}{Date \hfill Signature \hspace{5cm}}

\end{document}
